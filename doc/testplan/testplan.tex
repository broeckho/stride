\documentclass{article}

\usepackage{graphicx}
% Used for displaying a sample figure. If possible, figure files should
% be included in EPS format.
\usepackage{float}

\newcommand{\HRule}[1]{\rule{\linewidth}{#1}}

\setlength\parindent{24pt}

\begin{document}
\begin{titlepage}
	\begin{center}
		\HRule{0.5pt} \\
		\LARGE \textbf{\uppercase{Test plan}}
		\HRule{2pt} \\ [0.5cm]
		{\normalsize \today \par}
		\vspace{2cm}
		\begin{abstract}
			\noindent
			In this document we will go over all tests, including unit tests as well as scenario tests, documenting their name along with their purpose and how we intend to check this.
			
			%\keywords{First keyword  \and Second keyword \and Another keyword.}
		\end{abstract}
		\vfill
		\vfill
		{\normalsize Benjamin Vandersmissen \scriptsize benjamin.vandersmissen@student.uantwerpen.be \par}
		{\normalsize Lars Van Roy \scriptsize lars.vanroy@student.uantwerpen.be \par}
		{\normalsize Evelien Daems \scriptsize evelien.daems@student.uantwerpen.be \par}
		{\normalsize Frank Jan Fekkes \scriptsize franciscus.fekkes@student.uantwerpen.be \par}
	\end{center}
	
	\vfill
	
	\date{}
	
	\author{}
\end{titlepage}

\newpage

\tableofcontents

\newpage

\section{Introduction}
For the rest of this document we will give a short summary of every newly created test that will check if the new features work as we intend them to. These new additions include the two new contact types, being Daycare and PreSchool, new data formats, being JSON and HDF5 as well as global schemes that will be used by all students, a visual output for the data via QT and an improvement to the generation of the population by factoring in new parameters that will make the generation of the population more diverse, so that we can check even more parameters for their influence on the spread of diseases.

\section{Daycare and PreSchool}
\section{Data formats}
\subsection{JSON}
\subsection{HDF5}
\section{QT}
\section{improved population generation}
\subsection{demographic profile}
\subsubsection{genpop\_ages} 
\textbf{What will it test?} 
This test will check whether a given demographic profile for the young/old ratio is used. \\
\newline
\textbf{How will we achieve this?} 
We will achieve this by generating a population using a demographic profile for these ratios. We can check if the profile was used by comparing the generated population with the expected percentages.

\subsubsection{genpop\_cities}
\textbf{What will it test?} 
This test will check whether a given demographic profile for the distribution of central cities compared to other cities is used. \\
\newline
\textbf{How will we achieve this?}
We will achieve this by generating a population using a demographic profile for the different central cities versus other cities. We can check if the profile was used by comparing the generated population with the expected percentages.
\subsection{factory contactpools}

\end{document}