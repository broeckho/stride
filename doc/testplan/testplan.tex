\documentclass{article}

\usepackage{graphicx}
% Used for displaying a sample figure. If possible, figure files should
% be included in EPS format.
\usepackage{float}

\newcommand{\HRule}[1]{\rule{\linewidth}{#1}}

\setlength\parindent{24pt}

\begin{document}
\begin{titlepage}
	\begin{center}
		\HRule{0.5pt} \\
		\LARGE \textbf{\uppercase{Test plan}}
		\HRule{2pt} \\ [0.5cm]
		{\normalsize \today \par}
		\vspace{2cm}
		\begin{abstract}
			\noindent
			In this document we will go over all tests, including unit tests as well as scenario tests, documenting their name along with their purpose and how we intend to check this.
			
			%\keywords{First keyword  \and Second keyword \and Another keyword.}
		\end{abstract}
		\vfill
		\vfill
		{\normalsize Benjamin Vandersmissen \scriptsize benjamin.vandersmissen@student.uantwerpen.be \par}
		{\normalsize Lars Van Roy \scriptsize lars.vanroy@student.uantwerpen.be \par}
		{\normalsize Evelien Daems \scriptsize evelien.daems@student.uantwerpen.be \par}
		{\normalsize Frank Jan Fekkes \scriptsize franciscus.fekkes@student.uantwerpen.be \par}
	\end{center}
	
	\vfill
	
	\date{}
	
	\author{}
\end{titlepage}

\newpage

\tableofcontents

\newpage

\section{Introduction}
For the rest of this document we will give a short summary of every newly created test that will check if the new features work as we expected. These new additions include the two new contact types, being Daycare and PreSchool, new data formats, being JSON and HDF5 as well as global schemes that will be used by all students. There will also be a visual output for the data via QT and an improvement to the generation of the population by factoring in new parameters that will make the generation of the population more diverse, so that we can check even more parameters for their influence on the spread of diseases.

\section{Daycare and PreSchool}
\section{Data formats}
\subsection{JSON}
\subsection{HDF5}
\subsubsection{HDF5Reader.locationsTest}
\textbf{What will it test?}
This test will check whether the locations are read correctly from the HDF5 file. \\
\newline
\textbf{How will we achieve this?}
We will achieve this by comparing the locations read from the HDF5 input file with the correct locations.

\subsubsection{HDF5Reader.commutesTest}
\textbf{What will it test?}
This test will check whether the commutes are read correctly from the HDF5 file. \\
\newline
\textbf{How will we achieve this?}
We will achieve this by comparing the commutes read from the HDF5 input file with the correct commutes.

\subsubsection{HDF5Reader.contactCentersTest}
\textbf{What will it test?}
This test will check whether the contactCenters are read correctly from the HDF5 file. \\
\newline
\textbf{How will we achieve this?}
We will achieve this by comparing the contactCenters read from the HDF5 input file with the correct contactCenters.

\subsubsection{HDF5Reader.peopleTest}
\textbf{What will it test?}
This test will check whether the people are read correctly from the HDF5 file. \\
\newline
\textbf{How will we achieve this?}
We will achieve this by comparing the people read from the HDF5 input file with the correct people.

\subsubsection{HDF5Reader.invalidHDF5Test}
\textbf{What will it test?}
This test will check whether the HDF5Reader accepts invalid formed files or not. \\
\newline
\textbf{How will we achieve this?}
We will achieve this by feeding the HDF5Reader a number of invalid files and check whether or not an exception is thrown by the HDF5Reader.

\subsubsection{HDF5Writer.locationsTest}
\textbf{What will it test?}
This test will check whether the locations are written correctly to the HDF5 file. \\
\newline
\textbf{How will we achieve this?}
We will achieve this by comparing the resulting HDF5 file with the correct HDF5 file.

\subsubsection{HDF5Writer.commutesTest}
\textbf{What will it test?}
This test will check whether the commutes are written correctly to the HDF5 file. \\
\newline 
\textbf{How will we achieve this?}
We will achieve this by comparing the resulting HDF5 file with the correct HDF5 file.

\subsubsection{HDF5Writer.contactCentersTest}
\textbf{What will it test?}
This test will check whether the contactCenters are written correctly to the HDF5 file. \\
\newline
\textbf{How will we achieve this?}
We will achieve this by comparing the resulting HDF5 file with the correct HDF5 file.

\subsubsection{HDF5Writer.peopleTest}
\textbf{What will it test?}
This test will check whether the people are written correctly to the HDF5 file. \\
\newline
\textbf{How will we achieve this?}
We will achieve this by comparing the resulting HDF5 file with the correct HDF5 file.
\section{QT}
\section{Improved population generation}
\subsection{Demographic profile}
\subsubsection{DemographicProfile.age} 
\textbf{What will it test?} 
This test will check whether a given demographic profile for the young/old ratio is used. \\
\newline
\textbf{How will we achieve this?} 
We will achieve this by generating a population using a demographic profile for these ratios. We can check if the profile was used by comparing the generated population with the expected percentages.

\subsubsection{DemographicProfile.cities}
\textbf{What will it test?} 
This test will check whether a given demographic profile for the distribution of central cities compared to other cities is used. \\
\newline
\textbf{How will we achieve this?}
We will achieve this by generating a population using a demographic profile for the different central cities versus other cities. We can check if the profile was used by comparing the generated population with the expected percentages.
\subsection{Workplace contactpools}

\subsubsection{WorkplaceCSVReader.Test1}
\textbf{What will it test?}
This test will check whether we are able to read Workplace distribution CSV's.\\
\newline
\textbf{How will we achieve this?}
We will achieve this by passing a possible CSV configuration to the reader. The returned format can then be checked in order to ensure that the configuration is read correctly.

\subsubsection{WorkplaceSizePopulatorTest.NoPopulationTest}
\textbf{What will it test?}
The test will check whether the workplace size populator still works when there is no population available.\\
\newline
\textbf{How will we achieve this?}
We can easily check this by passing a geogrid to the populator who has a population size of 0. If there are no exception thrown we know that the function did finish correctly.

\subsubsection{WorkplaceSizePopulatorTest.OneWorkplaceTypeTest}
\textbf{What will it test?}
The test will check whether the workplace size populator can handle one single workplace type.\\
\newline
\textbf{How will we achieve this?}
We can achieve this by passing a single workplace size to populator. We can check if this process was successful by analyzing the altered object.

\subsubsection{WorkplaceSizePopulatorTest.TwoWorkplaceTypesTest}
\textbf{What will it test?}
The test will check whether the workplace size populator can handle two workplace types.\\
\newline
\textbf{How will we achieve this?} 
We can achieve this by passing two workplace sizes to the populator. In order to check if this was successful we can analyze the altered object by checking whether there are two different pools.

\subsubsection{WorkplaceSizePopulatorTest.MultipleWorkplaceTypesTest}
\textbf{What will it test?}
This test will check whether the workplace size populator can handle multiple types and sizes. \\
\newline
\textbf{How will we achieve this?}
We will achieve this by altering the configuration of the workplace sizes in such way that there are multiple. We can check if this was successful by analyzing the new populations data.

\subsubsection{WorkplaceSizeGeneratorTest.OneWorkplaceTypeTest}
\textbf{What will it test?}
This test will check whether the workplace size generator can handle one single type of workplace.\\
\newline
\textbf{How will we achieve this?}
We can achieve this by passing one single type to the generator. We can check whether this was successful by checking the resulting populations data.

\subsubsection{WorkplaceSizeGeneratorTest.ZeroWorkplaceTypesTest}
\textbf{What will it test?}
This test will check whether the workplace size generator can handle zero types of workplaces.\\
\newline
\textbf{How will we achieve this?}
We can achieve this by passing no types to the generator. To check if this was successful we can check whether the generated population has any workplace data.

\subsubsection{WorkplaceSizeGeneratorTest.FiveWorkplaceTypesTest}
\textbf{What will it test?}
This test will check whether the workplace size generator can handle multiple types of workplaces.\\
\newline
\textbf{How will we achieve this?}
We achieve this by passing five different types to the generator. The generated population will, if the process was successful, have five different workplaces, all with their expected sizes.

\end{document}